%--------------------------------------------------------------------
%--------------------------------------------------------------------
% Formato para los talleres del curso de Métodos Computacionales
% Universidad de los Andes
%--------------------------------------------------------------------
%--------------------------------------------------------------------

\documentclass[11pt,letterpaper]{exam}
\usepackage[utf8]{inputenc}
\usepackage[spanish]{babel}
\usepackage{graphicx}
\usepackage{tabularx}
\usepackage[absolute]{textpos} % Para poner una imagen en posiciones arbitrarias
\usepackage{multirow}
\usepackage{float}
\usepackage{hyperref}
%\decimalpoint

\begin{document}
\begin{center}
{\Large Métodos Computacionales} \\
Resultados DanielSabogal -2018\\
\end{center}


\noindent
\section{Gr\'aficas ejercicio 1}
Puede verse como la trayectoria de la particula tiene una aceleracion en el eje z nula y por el contrario en el eje x varia como en el eje y.
\begin{center}
\includegraphics[width=10cm]{ODE.pdf} 
\end{center}
x contra y, aqui puede verse como el campo productro cruz la velocidad genera una trayectoria circular en el plano z=0.
\begin{center}
\includegraphics[width=10cm]{ODE1.pdf}
\end{center}
x vs z, puede verse como z varia sinusioidalmente respecto a x tal como lo esperado en la teoria.
\begin{center}
\includegraphics[width=10cm]{ODE2.pdf}
\end{center}
t vs y, y varia respecto al tiempo sinusoidalmente.
\begin{center}
\includegraphics[width=10cm]{ODE3.pdf}
\end{center}

\section{Gr\'aficas ejercicio 2}
condiciones ininicales de la membrana.
\begin{center}
\includegraphics[width=10cm]{PDE.pdf}
\end{center}
membrana de extemos fijos en un tiempo final de 1/400 segundos, despues de ese tiempo la simulacion deja de funcionar sin razon alguna :(
\begin{center}
\includegraphics[width=10cm]{PDE1.pdf}
\end{center}
cortes transversales en la mitad de la membrana cada 50 pasos de tiempo, esto fue debido a que la simulacion solo funciono establemente 150 pasos de tiempo. Puede verse como la onda se propaga en dos dimensiones hacia la derecha.
\begin{center}
\includegraphics[width=10cm]{CORTES1.pdf}
\end{center}
membrana de extremos abiertos en un tiempo final de 1/400 segundos, al igual que la de extremos fijos, fue inestable despues de 150 iteraciones en el tiempo.
\begin{center}
\includegraphics[width=10cm]{PDE2.pdf}
\end{center}
puede verse que en los cortes transversales la membrana se propaga de la misma manera, sin embargo aparece un corte pronunciado en los bordes, esto se debe a la condicion de derivada espacial igual a cero, lo que hace que el punto anterior al final sea igual al final.
\begin{center}
\includegraphics[width=10cm]{CORTES2.pdf}
\end{center}
\end{document}
